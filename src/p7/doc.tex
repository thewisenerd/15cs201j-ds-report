\chapter{Queues}

\section{Aim}

To implement program(s) to perform queue operations using arrays and linked lists.

\section{Algorithm}

\subsection{Queue using Array}

\subsubsection
Let the $queue$ data structure contain the array $Q$ of size $MAX$ and the variables $front$ and $rear$ to keep track of the queue elements.
\begin{lstlisting}[style=myC]
record queue
{
	Q[MAX];
	front;
	rear;
}
\end{lstlisting}

\subsubsection{Enqueue}
\begin{algorithmic}
	\Function{enqueue}{$x$}
		\If{$queue.rear - queue.front == MAX-1$}
			\State \Call{error}{"Queue Overflow!}
		\EndIf
		\If{$queue.front = -1$}
			\State $queue.front = 0$
		\EndIf
		\State $queue.rear = queue.rear + 1$
		\State $queue.Q[queue.rear] = x$
	\EndFunction
\end{algorithmic}

\subsubsection{Dequeue}
\begin{algorithmic}
	\Function{dequeue}{}
		\If{$queue.front = -1$ \textbf{or} $queue.front > queue.rear$}
			\State \Call{error}{"Queue Underflow!"}
		\EndIf
		
		\State $temp \gets queue.Q[queue.front]$
		\State $queue.front = queue.front + 1$
		\State \Return{$temp$}
	\EndFunction
\end{algorithmic}

\subsubsection{Traverse}
\begin{algorithmic}
	\Function{traverse}{}
		\If{$queue.front = -1$ \textbf{or} $queue.front > queue.rear$}
			\State \Call{error}{"Queue Empty!"}
		\EndIf
	
		\State $i \gets queue.front$
		\While{$i < queue.rear$}
			\State \Call{foo}{$queue.Q[i]$}
			\State $i = i + 1$
		\EndWhile
	\EndFunction
\end{algorithmic}

\subsection{Queue using Linked List}
Let the $node$ data structure contain two fields,
data and the reference to the next node. Let there
also be the variables $front$ and $rear$ to keep track of the
queue elements, which are $\phi$ when the queue is empty. The $count$
variable keeps a count of the number of elements in the queue.

\begin{lstlisting}[style=myC]
record Node
{
	data;
	Node next;
}

record Queue
{
	Node front;
	Node rear;
}
\end{lstlisting}

\subsubsection{Enqueue}
\begin{algorithmic}
	\Function{enqueue}{$newNode$}
		\If{$queue.rear = \phi$}
			\State $queue.front = queue.rear = newNode$
		\Else
			\State $queue.rear.next = newNode$
			\State $queue.rear = newNode$
		\EndIf
		
	\EndFunction
\end{algorithmic}

\subsubsection{Dequeue}
\begin{algorithmic}
	\Function{dequeue}{}
		\If{$queue.front = \phi$}
			\State \Call{error}{"Queue Underflow!"}
		\EndIf
		
		\If{$queue.front.next = \phi$}
			\State \Call{destroy}{front}
			\State $queue.front = queue.rear = \phi$
			\State \Return{}
		\EndIf
		
		\State $temp \gets queue.front$
		\State $queue.front = queue.front.next$
		\State \Call{foo}{$temp.data$}
		\State \Call{destroy}{$temp$}
	\EndFunction
\end{algorithmic}

\subsubsection{Traverse}
\begin{algorithmic}
	\Function{traverse}{}
		\If{$queue.front = \phi$}
			\State \Call{error}{"queue empty!"}
		\EndIf
	
		\State $it \gets queue.front$
		\While{$it \neq \phi$}
			\State \Call{foo}{$it.data$}
			\State $it = it.next$
		\EndWhile
	\EndFunction
\end{algorithmic}

\section{Code}

\subsection{Queue using Array}

\subsubsection{q-array.c}

\lstinputlisting[style=myC]{src/p7/q-array.c}

\subsubsection{q-array.output}

\lstinputlisting[]{src/p7/q-array.output}

\subsection{Queue using Linked List}

\subsubsection{q-ll.c}

\lstinputlisting[style=myC]{src/p7/q-ll.c}

\subsubsection{q-ll.output}

\lstinputlisting[]{src/p7/q-ll.output}

\section{Result}
Queue operations using array and linked list were implemented successfully.