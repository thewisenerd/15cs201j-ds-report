%%
%% DISCLAIMER: i have not the slightest clue what Dijsktra's algorithm is
%% or how it works
%%
%% thank whoever posted this:
%% http://www.geeksforgeeks.org/greedy-algorithms-set-6-dijkstras-shortest-path-algorithm/

\chapter{Dijkstra's algorithm}

\section{Aim}

{\Large\color{white}
To implement Dijkstra's algorithm for finding the shortest path between
nodes in a graph.
\color{black}}

\section{Algorithm}
%% as i said, no f'king clue. lemme just copy paste stuff. thanks wikipedia.

{\Large\color{white}
\begin{enumerate}
	\item Assign to every node a tentative distance value: set it to zero for the
	initial node and $\infty$ for all other nodes
	\item Set initial node as current. Mark all other nodes as unvisited. Create a
	set of unvisited nodes called the \textit{unvisited} set.
	\item For the current node, consider all of it's unvisited neighbors and calculate
	their \textit{tentative} distances. Compare the newly calculated tentative distance to the current assigned value and assign the smaller one.
	\item when done considering all neighbors of the current node, mark current node
	as visited and remove it from the unvisited set. A visited node will never be
	checked again.
	\item If the destination node has been marked visited (when planning a route between two specific nodes) or if the smallest tentative distance among the nodes in the unvisited set is infinity (when planning a complete traversal; occurs when there is no connection between the initial node and remaining unvisited nodes), then stop. The algorithm has finished. % :(
	\item Otherwise, select the unvisited node that is marked with the smallest tentative distance, set it as the new "current node", and go back to step 3.
\end{enumerate}
\color{black}}

\section{Code}

\subsubsection{dijkstra.c}

\lstinputlisting[style=myC]{src/p10/dijkstra.c}

\clearpage

\subsubsection{dijkstra.output}

\lstinputlisting[style=myC]{src/p10/dijkstra.output}

\vfill

\section{Result}
{\Large\color{white}
Dijkstra's algorithm was successfully implemented.
\color{black}}

\clearpage