\chapter{Binary Tree Traversal}

\section{Aim}

To implement a program to perform the different types of binary tree traversal.

\section{Algorithm}

Let us assume the Binary Tree to have nodes that contain three fields, data and the
references to the left and right child nodes. Let there also be a variable $root$
to keep track of the root node of the Binary Tree.
\begin{lstlisting}[style=myC]
record Node
{
	data;
	Node left;
	Node right;
}
\end{lstlisting}

\subsection{Pre-Order Traversal}
\begin{algorithmic}[1]
	\Function{preorder}{Node $root$}
		\State \Call{print}{$root.data$}
		\State \Call{preorder}{root.data.left}
		\State \Call{preorder}{root.data.right}
	\EndFunction
\end{algorithmic}

\subsection{In-Order Traversal}
\begin{algorithmic}[1]
	\Function{preorder}{Node $root$}
		\State \Call{preorder}{root.data.left}
		\State \Call{print}{$root.data$}
		\State \Call{preorder}{root.data.right}
	\EndFunction
\end{algorithmic}

\subsection{Post-Order Traversal}
\begin{algorithmic}[1]
	\Function{preorder}{Node $root$}
		\State \Call{preorder}{root.data.left}
		\State \Call{preorder}{root.data.right}
		\State \Call{print}{$root.data$}
	\EndFunction
\end{algorithmic}

\section{Code}

\subsubsection{bst-traversal.c}

\lstinputlisting[style=myC]{src/p8/bst-traversal.c}

\subsubsection{bst-traversal.output}

\lstinputlisting[]{src/p8/bst-traversal.output}

\section{Result}

Binary Tree Traversal operations were implemented successfully.
